
\documentclass{resume} % Use the custom resume.cls style
\usepackage{xcolor}%定义了一些颜色
\usepackage{colortbl,booktabs}%第二个包定义了几个*rule
\usepackage{hyperref}
\usepackage[left=0.75in,top=0.6in,right=0.75in,bottom=0.6in]{geometry} % Document margins

\name{Yonggang Jiang} % Your name
\address{Tel: (+86) 181-7088-0522 $\diamond$ E-mail: \href{mailto:yonggangjiang@smail.nju.edu.cn}{yonggangjiang@smail.nju.edu.cn}}
\address{163 Xianlin Avenue, Qixia District, Nanjing, Jiangsu, China 210023}

% \address{Webpage: \url{ http://sunnyszy.me}}

\begin{document}

%----------------------------------------------------------------------------------------
%	EDUCATION SECTION
%----------------------------------------------------------------------------------------
\vspace{-2pt}
\begin{rSection}{Education}
{\bf No.3 Middle School of Nanchang} \hfill{\em Sep. 2014 - Jul. 2017}\\
{\bf Nanjing University} \hfill {\em Sep. 2017 - Jul. 2021(expected)} \\
B.C. in Computer Science and Technology, Elite Program \\
GPA: 94.4/100 {\em (Major)} ~ 92.4/100 {\em (Overall)}\\
Ranking: 1/31 {\em (Major)} ~ 3/204 {\em (Overall)}\smallskip\\
{\bf University of California, Berkeley} \hfill {\em Jan. 2020 - May. 2020} \\
Berkeley International Study Program
\smallskip\\
GPA: 4.0/4.0
\end{rSection}

%\begin{rSection}{Publications}
%$[1]$ X. Tong, {\bf Jiachen Sun}, Xiaohua Tian. ``{\bf FineLoc: A Fine-grained Self-calibrating Wireless Indoor Localization System}'' submitted to {\em Infocom}. 2018.\medskip\\
%$[2]$ {\bf Jiachen Sun}$^\dagger$, Y. Zhang$^\dagger$, Y. Yao$^\dagger$. ``{\bf Replacing the Kariba Dam}'' {\em UMAP Journal of Undergraduate Mathematics and Its Applications}. 2017. $^\dagger$Co-primary Authors
%\end{rSection}
\begin{rSection}{Research Interests}
I have broad interests in theoretical computer science, especially in sampling and distributed algorithms. Currently, I am focusing on classical distributed problems like Byzantine consensus. I am also considering problems related to sampling using spatial mixing property.
\end{rSection}
\begin{rSection}{Research Experience}

{\large Research work in TCS group at {\bf Nanjing University} }

\begin{rSubsection}{Boosting Distributed Las Vegas Algorithms}{\em Jul. 2019 - Oct. 2020}{(Advisor: \href{http://tcs.nju.edu.cn/yinyt/}{\bf Prof. Yitong Yin})}{}
\item The Las Vegas algorithm has two  definitions:  bounded time with  checkable failure and random time that is always correct.  The difference in the two definitions is trivial in the computation model with a central controller but non-trivial in a distributed model. We used an algorithm to implement the reduction in the LOCAL model.
\item Our algorithm can also be applied to the uniform sampling LLL (Lovasz Local Lemma) instance.  The expected run time is polynomial on log n when the instance has a constant, correct rate. Moreover, the expected run time of our algorithm is nearly exponentially convergent, i.e., highly concentrated in its expectation.
\end{rSubsection}
\begin{rSubsection}{Current work in Distributed Computing}{\em Aug. 2020 - Present}{ (Advisor: \href{https://chaodong.me/}{\bf Prof. Chaodong Zheng})}{}
\item Byzantine consensus protocol with recourse competitive analysis.
\item Contention resolution problem with jamming and without collision detection.
\end{rSubsection}
\begin{rSubsection}{Dynamic Sampling}{\em Jan. 2019 - Jun. 2019}{ (Advisor: \href{http://tcs.nju.edu.cn/yinyt/}{\bf Prof. Yitong Yin})}{}
\item Based on previous work {\em Dynamic Sampling from Graphical Models}.
\item Improve the condition for the fast convergence of the dynamic sampling hardcore model.
\end{rSubsection}
\end{rSection}
\begin{rSection}{Teaching experiences}
\textbf{Data Structures and Algorithms, Nanjing University, Fall 2019} \hfill{\em Sep. 2019 - Jan. 2020}\\
\textbf{Data Structures and Algorithms, Nanjing University, \href{https://chaodong.me/teaching/dsalg/2020/course-homepage.html}{Fall 2020}} \hfill{\em Sep. 2020 - Present}\\
\end{rSection}
\begin{rSection}{Key Projects}

\begin{rSubsection}{Predicting the Run Time of Sklearn Programs}{\em Jan. 2019 - Feb 2019}{
{\large Group work at Hong Kong University of Science and Technology}}{}

\item Predicted the run time of a certain function on some given parameters, based on given samples.
\item Used k-nearest neighbor (KNN) method, yeilding a final result ranked third among all participants.
\end{rSubsection}
\begin{rSubsection}{Digital Circuit and System Final Project}{\em Jan. 2019}{Joint work with Yangbo Zhang and Xingliang Du}{ Project leader}
\item Implemented a multi-cycle Mips instruction architecture CPU.
\item Functions included typing into the shell and executing some simple operations (lighting, clearing the screen, calculating simple expressions, etc.)
\end{rSubsection}
\begin{rSubsection}{Introduction to Computer Systems (ICS) Project}{\em Sep. 2018 - Jan. 2019}{}{}
\item Achieved a full-featured N86 simulator NEMU (NJU emulator) and ran the game "Legend of Swordsman".
\end{rSubsection}
\end{rSection}

\begin{rSection}{Honors \& Awards}
National Elite Program Scholarship, Outstanding Prize (top 1)\hfill{\em 2019-2020}\\
National Elite Program Scholarship, Outstanding Prize (top 1)\hfill{\em 2018-2019}\\
National Elite Program Scholarship, First Prize (top 1)\hfill{\em 2017-2018}\\
CCPC Regional Contest, Gold Medal: Guilin \hfill{\em 2018}\\
ACM-ICPC Asia Regional Contest, Silver Medal: Jiaozuo \hfill{\em 2018}\\
NOIP First Prize \hfill{\em 2015}\\
\end{rSection}
\begin{rSection}{Skills}

\begin{tabular}{ @{} >{\bfseries}l @{\hspace{2ex}} l }
Computer Languages & C/C++, Java, SQL\\
English & TOEFL({\em MyBest™}):
104 (Reading 29, Listening 29, Speaking 23, Writing 23)\\
GRE & 153 (Verbal 153, Quantitative 170, Writing 3.5)
\end{tabular}
\end{rSection}
\end{document}
